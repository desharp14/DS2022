\documentclass[a4paper]{article}
\usepackage[utf8]{ctex}
\usepackage{listings}

\title{作业八:Bellman-Ford 算法的实现}
\author{陈俊铭 3210300364 (信息与计算科学)}
\date{\today}

\begin{document}
\maketitle
\pagenumbering{gobble}
\section{设计思路}
Bellman-Ford算法是一种处理存在 负权边的单元最短路问题的算法。

测试步骤:
\begin{itemize}
\item 通过Bellman-Ford 算法,检查正确性
\item 测试是否有负权循
\item 测试试运行时间和复杂度
\end{itemize}

\section{测试结果}
\noindent Bellman Ford:\\
- delta(s,s)0\\
- delta(s,t)2\\
- delta(s,x)4\\
- delta(s,y)7\\
- delta(s,z)-2\\

\noindent Negative weighted detected.\\

\noindent Test complexity(s):\\
|V|         500         1000        2000        4000\\
|E|= 2*|V|  0.011       0.038       0.159       0.736\\
|E|= 4*|V|  0.017       0.061       0.249       1.14\\
|E|= 8*|V|  0.026       0.108       0.497       2.466\\
|E|= 16*|V| 0.072       0.25        1.157       6.199\\

\end{document}