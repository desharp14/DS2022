\documentclass[a4paper]{article}
\usepackage[utf8]{ctex}
\usepackage{listings}

\title{作业四:二叉树和节点的逻辑设计}
\author{陈俊铭 3210300364 (信息与计算科学)}
\date{\today}

\begin{document}
\maketitle
\section*{设计思路}
Node是一种数据结构,是存储一个值的各种不同节点的基础类。Node 类的实现应该具有获取存储在 Node 中的值、获取下一个节点以及设置到下一个节点的链接的方法。

BinaryTree 是一种非线性数据结构,其中每个节点最多只有两个分支(即不存在分支度大于2的节点)的树结构。BinaryTree 的函数包括构造,析构函数,实现找最大最小元素,判断元素是否存在树中,打印,清空树,判断树是否为空,以及插入和删除功能。

BinarySearchTree 是具有结构化节点组织的 Binary Tree,主要用于排序、检素和搜索数据的树。BinarySearchTree 自然继承 BinaryTree 的所有函数,同时新增 BinarySearchTree 特有的函数如 findMax()、findMin() 等,

AvlTree 是一种自平衡的 BinaryTree,主要控制 BinaryTree 的高度。类似,AvlTree 继承 BinaryTree 的所有函数,同时新增 BinarySearchTree 特有的函数,如rotate()、balance()和 height()。

\end{document}