\documentclass[a4paper]{article}
\usepackage[utf8]{ctex}
\usepackage{listings}

\title{作业六:排序算法类的实现}
\author{陈俊铭 3210300364 (信息与计算科学)}
\date{\today}

\begin{document}
\maketitle
\section{设计思路}
使用和比较两种排序算法,快速排序 (quicksort) 和堆排序 (heapsort)。

\section{测试结果}
\noindent When n = 10000,\\
- efficiency: 0.01, heaptime: 0.003 s, quicktime: 0 s.\\
- efficiency: 0.1, heaptime: 0.006 s, quicktime: 0.001 s.\\
- efficiency: 0.9, heaptime: 0.005 s, quicktime: 0.001 s.\\
- efficiency: 0.99, heaptime: 0.003 s, quicktime: 0.001 s.\\

\noindent When n = 100000,\\
- efficiency: 0.01, heaptime: 0.035 s, quicktime: 0.005 s.\\
- efficiency: 0.1, heaptime: 0.028 s, quicktime: 0.005 s.\\
- efficiency: 0.9, heaptime: 0.022 s, quicktime: 0.006 s.\\
- efficiency: 0.99, heaptime: 0.034 s, quicktime: 0.006 s.\\

\noindent When n = 1000000,\\
- efficiency: 0.01, heaptime: 0.286 s, quicktime: 0.065 s.\\
- efficiency: 0.1, heaptime: 0.284 s, quicktime: 0.053 s.\\
- efficiency: 0.9, heaptime: 0.274 s, quicktime: 0.051 s.\\
- efficiency: 0.99, heaptime: 0.27 s, quicktime: 0.056 s.\\

\section{快排改进}
当输入包含太多相同元素时,我们可以对其进行润色。选择主元,或围绕其划分列表的元素,是快速排序中的关键操作之一。快速排序中的枢轴元素通常是分区的最外层元素之一,要么在左边,要么在右边。对于已排序或接近已排序的输入,此选择将导致最坏的情况。通过选择主元分区的第一个、中间的和最后一个元素的中位数,或者为主元选择一个随机索引,我们可以快速高效地解决问题( median3)。

\end{document}